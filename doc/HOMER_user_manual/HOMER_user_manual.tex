% HOMER User Manual
%
% Please note this document will be automatically compiled and hosted online
% after each commit to master. Because of this, renaming or moving the
% document should be done carefully. To see the compiled document, go to
% http://planets.ucf.edu/bart-docs/HOMER_user_manual.pdf

\documentclass[letterpaper, 12pt]{article}
\input{top-HOMER_user_manual}

\begin{document}

\begin{titlepage}
\begin{center}

\textsc{\LARGE University of Central Florida}\\[1.5cm]

% Title
\rule{\linewidth}{0.5mm} \\[0.4cm]
{ \huge \bfseries HOMER Users Manual \\[0.4cm] }
\rule{\linewidth}{0.5mm} \\[1.0cm]

\textsc{\Large Helper Of My Eternal Retrievals}\\[1.5cm]

% Author and supervisor
\noindent
\begin{minipage}{0.4\textwidth}
\begin{flushleft}
\large
\emph{Authors:} \\
Michael D. \textsc{Himes} \\
\end{flushleft}
\end{minipage}%
\begin{minipage}{0.4\textwidth}
\begin{flushright} \large
\emph{Supervisor:} \\
Dr.~Joseph \textsc{Harrington}
\end{flushright}
\end{minipage}
\vfill

% Bottom of the page
{\large \today}

\end{center}
\end{titlepage}

\tableofcontents
\newpage

\section{Team Members}
\label{sec:team}

\begin{itemize}
\item \href{https://github.com/mdhimes/}{Michael Himes}%
  \footnote{https://github.com/mdhimes/}, University of
  Central Florida (mhimes@knights.ucf.edu)
\item Joseph Harrington, University of Central Florida
\item David C. Wright, University of Central Florida
\item Zacchaeus Scheffer, University of Central Florida
\end{itemize}

\section{Introduction}
\label{sec:theory}

\noindent This document describes HOMER, the Helper Of My Eternal Retrievals.
HOMER retrieves atmospheric properties using a neural network (NN) model of 
radiative transfer (RT), such as that trained by MARGE.  HOMER uses the 
Multi-Core Markov Chain Monte Carlo (MC3) code to explore a parameter space 
and determine a posterior distribution of models consistent with the observed 
data.

The detailed HOMER code documentation and User Manual are provided 
with the package to assist users in its usage. For additional support, contact 
the lead author (see Section \ref{sec:team}).

HOMER is released under the Reproducible Research Software License.  
For details, see \\
\href{https://planets.ucf.edu/resources/reproducible-research/software-license/}{https://planets.ucf.edu/resources/reproducible-research/software-license/}.
\newline

\noindent The HOMER package is organized as follows: \newline
% The framebox and minipage are necessary because dirtree kills the
% indentation.
\noindent\framebox{\begin{minipage}[t]{0.97\columnwidth}%
\dirtree{%
 .1 HOMER. 
 .2 code.
 .3 MCcubed.
 .4 MCcubed.
 .5 mc.
 .5 rednoise.
 .2 doc.
 .2 example.
 .2 lib. 
 .2 modules. 
 .3 MCcubed. 
}
\end{minipage}}
\vspace{0.7cm}
% \newline is not working here, therefore I use vspace.
% (because dirtree is such a pain in the ass)

\section{Installation}
\label{sec:installation}

\subsection{System Requirements}
\label{sec:requirements}

\noindent HOMER was developed on a Unix/Linux machine using the following 
versions of packages:

\begin{itemize}
\item Python 3.7.2
\item GPyOpt 1.2.5
\item Keras 2.2.4
\item Numpy 1.16.2
\item Matplotlib 3.0.2
\item Scipy 1.2.1
\item sklearn 0.20.2
\item Tensorflow 1.13.1
\item CUDA 9.1.85
\item cuDNN 7.5.00
\item gcc 7.5.0
\end{itemize}

This conveniently matches the dependencies of MARGE.



\subsection{Install and Compile}
\label{sec:install}

\noindent To begin, obtain the latest stable version of HOMER.  

\noindent First, create a local directory to hold HOMER.  Let the path to this 
directory be \texttt{`}localHOMERdir\texttt{`}.

\begin{verbatim}
mkdir `localHOMERdir`
cd `localHOMERdir`
\end{verbatim}

\noindent Now, clone the repository:
\begin{verbatim}
git clone --recursive https://github.com/exosports/HOMER .
\end{verbatim}

\noindent Now, modify MC3 as necessary:

\begin{verbatim}
make mccubed
\end{verbatim}

\noindent HOMER contains a file to easily build a conda environment capable of 
executing the software.  Create the environment via

\begin{verbatim}
conda env create -f environment.yml
\end{verbatim}

\noindent Then, activate the environment:

\begin{verbatim}
conda activate homer
\end{verbatim}

\noindent If the user wishes to compute median and 1-2-3sigma spectra, 
the datasketches package is required.  Add this to the environment via
\begin{verbatim}
make datasketches
\end{verbatim}
\noindent from the HOMER directory. Note that this package is NOT required 
to run HOMER, it is only required to calculate spectra quantiles.

\noindent You are now ready to run HOMER!


\section{Example}
\label{sec:example}

This directory holds an example of how to run HOMER. It matches the use case 
demonstrated in Himes et al. (2020).

\noindent \textbf{NOTE}: Executing this example requires files from the 
MARGE example!  Run that first if you wish to run this example.\newline

\noindent The recommend system specs to run the HOMER example are less than 
those of the MARGE example.  For users with limited RAM (4 GB or less), the 
total number of iterations may need to be reduced.  For system storage, 
{\textless}1 GB is required.  Without a GPU, it will take around 1 day 
to execute, compared to less than an hour with a modern GPU.\newline

\noindent If using an operating system that is not Linux-based, some aspects of 
the example will likely need to be adjusted.  Users are encouraged to submit 
updates to this example guide via pull requests if they find modifications are 
necessary for certain operating systems.


\subsection{Walkthrough}

Ensure that the repo's submodules have also been cloned.  
When cloning HOMER, this can be done by
\begin{verbatim}
git clone --recursive https://github.com/exosports/HOMER HOMER/
cd MARGE/
\end{verbatim}
Alternatively, if HOMER has already been cloned, pull the submodules by 
navigating to the HOMER directory and 
\begin{verbatim}
git submodule init
git submodule update
\end{verbatim}

\noindent Next, build MCcubed:
\begin{verbatim}
make mccubed
\end{verbatim}

\noindent Now, we are ready to begin.\newline

\noindent Navigate to the directory where the example files are at.
\begin{verbatim}
cd example
\end{verbatim}
If the user has copied these files to another location outside of HOMER, 
navigate there instead.  The paths in some input files will need to be changed 
by the user. \newline

\noindent Execute HOMER:
\begin{verbatim}
../HOMER.py example.cfg
\end{verbatim}

\noindent It will begin the retrieval, and should take \textless 1 hour.



\section{Program Inputs}
\label{sec:inputs}

The executable HOMER.py is the driver for the HOMER program. It takes a 
a configuration file of parameters.  Once configured, HOMER is executed via 
the terminal as described in Section \ref{sec:example}.


\subsection{HOMER Configuration File}
\label{sec:config}
The HOMER configuration file is the main file that sets the arguments for a 
HOMER run. The arguments follow the format {\ttb argument = value}, where 
{\ttb argument} is any of the possible arguments described below. 

\noindent The available options for a HOMER configuration file are listed below.

\noindent \underline(Directories)
\begin{itemize}
\item inputdir   : str.  Directory containing HOMER inputs.
\item outputdir  : str.  Directory containing HOMER outputs.
\end{itemize}

\noindent \underline{Run Parameters}
\begin{itemize}
\item onlyplot    : bool. Determines whether to skip the MCMC.
                    Reproduces plots \& calculations related to posterior.
\item credregion  : bool. Determines whether to calculate the 68, 95, and 99\% 
                    credible regions \& uncertainties.
\item compost     : bool. Determines whether to compare HOMER's posterior to 
                          another.
\item compfile    : str.  Path to posterior to compare with HOMER.
\item compname    : str.  Name of the other posterior for plot legends.
\item compsave    : str.  File name prefix for the saved comparison plots.
\item compshift   : floats. Shifts all values of a particular parameter by a 
                          set amount in the posterior to be compared, such as 
                          for unit conversions.
                    Format: val1 val2 val3 val4 ...
\item postshift   : floats. Same as \texttt{`}compshift\texttt{`}, but for 
                            HOMER's posterior.
\end{itemize}


\noindent \underline{Data Normalization Parameters}
\begin{itemize}
\item ilog        : bool. Determines whether the NN takes the logarithm of the 
                          inputs.
\item olog        : bool. Determines whether the NN predicts the log of the 
                          outputs.
\item normalize   : bool. Determines whether to standardize the data by its 
                          mean and standard deviation.
\item scale       : bool. Determines whether to scale the data to be within a 
                          range.
\item scalelims   : ints. Range to scale the data to.
                          Format: low, high
\item fmean       : str.  Path to .NPY file of mean training input and output 
                          values.
                          Format: [inp0, inp1, ..., outp0, outp1, ...]
                          If relative path, assumed to be with respect to the 
                          input directory.
\item fstdev      : str.  Path to .NPY file of standard deviation of 
                          inputs/outputs.
                          See \texttt{`}fmean\texttt{`} for format \& path 
                          description.
\item fmin        : str.  Path to .NPY file of minima of inputs/outputs.
                          See \texttt{`}fmean\texttt{`} for format \& path 
                          description.
\item fmax        : str.  Path to .NPY file of maxima of inputs/outputs.
                          See \texttt{`}fmean\texttt{`} for format \& path 
                          description.
\end{itemize}


\noindent \underline{Neural Network (NN) Parameters}
\begin{itemize}
\item weight\_file: str.  File containing NN model weights.
                   NOTE: MUST end in .h5
\item input\_dim  : int.  Dimensionality of the input  to the NN.
\item output\_dim : int.  Dimensionality of the output of the NN.
\item convlayers : ints. Number of nodes for each convolutional layer.
\item denselayers: ints. Number of nodes for each dense         layer.
\end{itemize}

\noindent \underline{MCMC Parameters}
\begin{itemize}
\item flog        : str.  Path to MCMC log file. 
                          If relative, with respect to input dir.
\item func        : strs. Function and file to evaluate at each iteration of 
                          the MCMC.
                          Format: function file
                          Note: omit the '.py' from \texttt{`}file\texttt{`}.
\item pnames      : strs. Name of each free parameter. Can include LaTeX 
                          formatting.
\item pinit       : floats. Initial parameters for the MCMC.
\item pmin        : floats. Minima for free parameters.
\item pmax        : floats. Maxima for free parameters.
\item pstep       : floats. Step size for free parameters. 
                          This will change throughout the MCMC due to the 
                          differential evolution algorithm used.
\item niter       : int.  Number of total iterations.
\item burnin      : int.  Number of burned iterations from the beginning of 
                          chains.
\item nchains     : int.  Number of parallel samplers.
\item thinning    : int.  Thinning factor for posterior.
\item data        : floats. Values to be fit via MCMC. 
                          Format: Separate each value by an indented new line.
\item uncert      : floats. Uncertainties on values to be fit via MCMC. 
                          Same format.
\item filters     : strs. Paths to filters associated with each datum. 
                          Same format.
\item starspec    : str.  Path to .NPY file of the stellar spectrum.
\item factor      : str.  Path to .NPY file of multiplication factor by which 
                          to modify de-normalized predictions. 
                          E.g., unit conversion.
\item wnfact      : float. Multiplication factor to convert 
                           \texttt{`}xvals\texttt{`} to cm-1.
\item filt2um     : float. Multiplication factor to convert the filter 
                          wavelengths to microns.
\item PTargs      : str.  Path to .txt file containing values necessary to 
                          calculate the temperature--pressure profile.
                          Currently, only option is Line et al. (2013) method.
                          Format: R\_star (m), T\_star (K), T\_int (K), 
                                  sma    (m), grav (cm s-2)

\end{itemize}

\noindent \underline{Plotting Parameters}
\begin{itemize}
\item xvals      : str.  Path to .NPY file containing the x-axis values 
                         associated with a prediction.
                         If relative, path is with respect to 
                         \texttt{`}inputdir\texttt{`}.
\item xval\_label : str.  X-axis label for plots.
\item yval\_label : str.  Y-axis label for plots.
\item fpress     : str.  Path to text file containing the pressures of each 
                         layer of the atmosphere, for plotting T(p) profiles.
                         If relative, with respect to 
                         \texttt{`}inputdir\texttt{`}.
\item savefile   : str.  Prefix for MCMC plots to be saved.
\end{itemize}



\section{Program Outputs}
\label{sec:outputs}

HOMER produces the following outputs:

\begin{itemize}
\item MCMC.log -- a record of the MCMC
\item output.npy -- the posterior determined by the MCMC
\item MCMC plots -- pairwise, posterior, PT, and trace plots.
\item ess.txt -- the effective sample size (ESS) of the run.
\item credregion.txt -- the 68, 95, and 99\% credible regions.
\item comparison plots -- if \texttt{`}compost\texttt{`}, makes plots of the 
                          1D marginalized 
                          posteriors, 2D pairwise posteriors, and explored 
                          temperature--presure profiles.
\item bhatchar.npy -- if \texttt{`}compost\texttt{`}, calculates the 
                      Bhattacharyya coefficients between HOMER's posterior and 
                      the other specified posterior.
\end{itemize}



\section{Be Kind}
\label{sec:bekind}
Please cite this paper if you found this package useful for your
research:

\begin{itemize}
\item Himes et al. (2020), submitted to PSJ.
\end{itemize}

\begin{verbatim}
@article{HimesEtal2020psjMARGEHOMER,
   author = {{Himes}, Michael D. and {Harrington}, Joseph and {Cobb}, Adam D. and {G{\"u}ne{\textcommabelow s} Baydin}, At{\i}l{\i}m and {Soboczenski}, Frank and
         {O'Beirne}, Molly D. and {Zorzan}, Simone and
         {Wright}, David C. and {Scheffer}, Zacchaeus and
         {Domagal-Goldman}, Shawn D. and {Arney}, Giada N.},
    title = "Accurate Machine Learning Atmospheric Retrieval via a Neural Network Surrogate Model for Radiative Transfer",
  journal = {PSJ},
     year = 2020,
    pages = {submitted to PSJ}
}
\end{verbatim}

\noindent Thanks!

% \section{Further Reading}
% \label{sec:furtherreading}

% TBD: Add papers here.


\end{document}

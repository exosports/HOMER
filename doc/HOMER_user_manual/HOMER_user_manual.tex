% HOMER User Manual
%
% Please note this document will be automatically compiled and hosted online
% after each commit to master. Because of this, renaming or moving the
% document should be done carefully. To see the compiled document, go to
% https://exosports.github.io/HOMER/doc/HOMER_User_Manual.html

\documentclass[letterpaper, 12pt]{article}
\input{top-HOMER_user_manual}

\begin{document}

\begin{titlepage}
\begin{center}

\textsc{\LARGE University of Central Florida}\\[1.5cm]

% Title
\rule{\linewidth}{0.5mm} \\[0.4cm]
{ \huge \bfseries HOMER Users Manual \\[0.4cm] }
\rule{\linewidth}{0.5mm} \\[1.0cm]

\textsc{\Large Helper Of My Eternal Retrievals}\\[1.5cm]

% Author and supervisor
\noindent
\begin{minipage}{0.4\textwidth}
\begin{flushleft}
\large
\emph{Authors:} \\
Michael D. \textsc{Himes} \\
\end{flushleft}
\end{minipage}%
\begin{minipage}{0.4\textwidth}
\begin{flushright} \large
\emph{Supervisor:} \\
Dr.~Joseph \textsc{Harrington}
\end{flushright}
\end{minipage}
\vfill

% Bottom of the page
{\large \today}

\end{center}
\end{titlepage}

\tableofcontents
\newpage

\section{Team Members}
\label{sec:team}

\begin{itemize}
\item \href{https://github.com/mdhimes/}{Michael Himes}%
  \footnote{https://github.com/mdhimes/}, University of
  Central Florida (mhimes@knights.ucf.edu)
\item Joseph Harrington, University of Central Florida
\item David C. Wright, University of Central Florida
\item Zacchaeus Scheffer, University of Central Florida
\end{itemize}

\section{Introduction}
\label{sec:theory}

\noindent This document describes HOMER, the Helper Of My Eternal Retrievals.
HOMER retrieves atmospheric properties using a neural network (NN) model of 
radiative transfer (RT), such as that trained by MARGE.  HOMER uses the 
Multi-Core Markov Chain Monte Carlo (MC3) code to explore a parameter space 
and determine a posterior distribution of models consistent with the observed 
data.

The detailed HOMER code documentation and User Manual\footnote{Most recent version of the manual available at 
\href{https://exosports.github.io/HOMER/doc/HOMER_User_Manual.html}{https://exosports.github.io/HOMER/doc/HOMER\_User\_Manual.html}} 
are provided with the package to assist users in its usage. 
For additional support, contact the lead author (see Section \ref{sec:team}).

HOMER is released under the Reproducible Research Software License.  
For details, see \\
\href{https://planets.ucf.edu/resources/reproducible-research/software-license/}{https://planets.ucf.edu/resources/reproducible-research/software-license/}.
\newline

\noindent The HOMER package is organized as follows: \newline
% The framebox and minipage are necessary because dirtree kills the
% indentation.
\noindent\framebox{\begin{minipage}[t]{0.97\columnwidth}%
\dirtree{%
 .1 HOMER. 
 .2 code.
 .3 MCcubed.
 .4 MCcubed.
 .5 mc.
 .5 rednoise.
 .2 doc.
 .2 example.
 .2 lib. 
 .2 modules. 
 .3 MCcubed. 
}
\end{minipage}}
\vspace{0.7cm}
% \newline is not working here, therefore I use vspace.
% (because dirtree is such a pain in the ass)

\section{Installation}
\label{sec:installation}

\subsection{System Requirements}
\label{sec:requirements}

\noindent HOMER was developed on a Linux machine using the following 
versions of packages:

\begin{itemize}
\item Python 3.7.2
\item Keras 2.2.4
\item Numpy 1.16.2
\item Matplotlib 3.0.2
\item Scipy 1.2.1
\item sklearn 0.20.2
\item Tensorflow 1.13.1
\item CUDA 9.1.85
\item cuDNN 7.5.00
\end{itemize}

\noindent This conveniently matches the dependencies of MARGE.

\noindent If installing the Datasketches library, CMake 3.12.0+ is required.



\subsection{Install and Compile}
\label{sec:install}

\noindent To begin, obtain the latest stable version of HOMER.  

\noindent First, decide on a local directory to hold HOMER.  Let the path to this directory 
be `HOMER'.  Now, clone the repository:
\begin{verbatim}
git clone --recursive https://github.com/exosports/HOMER HOMER/
cd HOMER/
\end{verbatim}

\noindent HOMER contains a file to easily build a conda environment capable of 
executing the software.  Create the environment via

\begin{verbatim}
conda env create -f environment.yml
\end{verbatim}

\noindent Then, activate the environment:

\begin{verbatim}
conda activate marge_homer
\end{verbatim}

\noindent Now, build the submodules:

\begin{verbatim}
make all
\end{verbatim}

\noindent The Datasketches library is optional.  If you do not want to install it, do 
\begin{verbatim}
make mccubed
\end{verbatim}
\noindent instead of `make all'.

\noindent You are now ready to run HOMER.


\section{Example}
\label{sec:example}

The following script will walk a user through using HOMER for a retrieval on 
HD 189733 b, following from MARGE's example which trains an NN model for RT. 
These instructions are meant to be executed from a Linux terminal.  
The requirements for this example are less than the requirements for MARGE's 
example.  Since MARGE's example is required to execute this example, we do 
not explicitly list all system requirements.  Ensure you have at least 1 GB of 
free space before beginning.

\noindent To begin, copy the requisite files to a directory parallel to HOMER. 
Beginning from HOMER/, 
\begin{verbatim}
mkdir ../run
cp -a ./example/* ../run/.
cd ../run
\end{verbatim}
\noindent Note that, if the user is running this following the MARGE example, 
the `run' directory should contain all of the files from both examples.

\noindent Now, execute HOMER:

\begin{verbatim}
../HOMER/HOMER.py HOMER.cfg
\end{verbatim}

\noindent This will take some time to run. 


\section{Program Inputs}
\label{sec:inputs}

The executable HOMER.py is the driver for the HOMER program. It takes a 
a configuration file of parameters.  Once configured, HOMER is executed via 
the terminal as described in Section \ref{sec:example}.


\subsection{HOMER Configuration File}
\label{sec:config}
The HOMER configuration file is the main file that sets the arguments for a 
HOMER run. The arguments follow the format {\ttb argument = value}, where 
{\ttb argument} is any of the possible arguments described below. 

\noindent The available options for a HOMER configuration file are listed below.

\noindent \underline(Directories)
\begin{itemize}
\item inputdir   : str.  Directory containing HOMER inputs.
\item outputdir  : str.  Directory containing HOMER outputs.
\end{itemize}

\noindent \underline{Run Parameters}
\begin{itemize}
\item quantiles   : bool. Determines whether to compute spectra quantiles.
                    If the Datasketches library is not install, this setting 
                    has no effect.
\item onlyplot    : bool. Determines whether to skip the MCMC.
                    Reproduces plots \& calculations related to posterior.
\item plot\_PT    : bool. Determines whether to compute the explored 
                    pressure--temperature profiles using the formulation of 
                    Line et al. (2013)
\item credregion  : bool. Determines whether to calculate the 68, 95, and 99\% 
                    credible regions \& uncertainties.
\item compost     : bool. Determines whether to compare HOMER's posterior to 
                          another.
\item compfile    : str.  Path to posterior to compare with HOMER.
\item compname    : str.  Name of the other posterior for plot legends.
\item compsave    : str.  File name prefix for the saved comparison plots.
\item compshift   : floats. Shifts all values of a particular parameter by a 
                          set amount in the posterior to be compared, such as 
                          for unit conversions.
                    Format: val1 val2 val3 val4 ...
\item postshift   : floats. Same as `compshift`, but for HOMER's posterior.
\end{itemize}


\noindent \underline{Data Normalization Parameters}
\begin{itemize}
\item ilog        : bool. Determines whether the NN takes the logarithm of the 
                          inputs.
                          Alternatively, specify comma-, space-, or newline-separated 
                          integers to selectively take the log of certain inputs.
\item olog        : bool. Determines whether the NN predicts the log of the 
                          outputs.
                          Alternatively, specify comma-, space-, or newline-separated 
                          integers to selectively take the log of certain outputs.
\item normalize   : bool. Determines whether to standardize the data by its 
                          mean and standard deviation.
\item scale       : bool. Determines whether to scale the data to be within a 
                          range.
\item scalelims   : ints. Range to scale the data to.
                          Format: low, high
\item fmean       : str.  Path to .NPY file of mean training input and output 
                          values.
                          Format: [inp0, inp1, ..., outp0, outp1, ...]
                          If relative path, assumed to be with respect to the 
                          input directory.
\item fstdev      : str.  Path to .NPY file of standard deviation of 
                          inputs/outputs.
                          See `fmean` for format \& path description.
\item fmin        : str.  Path to .NPY file of minima of inputs/outputs.
                          See `fmean` for format \& path description.
\item fmax        : str.  Path to .NPY file of maxima of inputs/outputs.
                          See `fmean` for format \& path description.
\end{itemize}


\noindent \underline{Neural Network (NN) Parameters}
\begin{itemize}
\item weight\_file: str.  File containing NN model and weights.
                    NOTE: MUST end in .h5
\item inD  : int.  Dimensionality of the input  to the NN.
\item outD : int.  Dimensionality of the output of the NN.
\end{itemize}

\noindent \underline{MCMC Parameters}
\begin{itemize}
\item flog        : str.  Path to MCMC log file. 
                          If relative, with respect to input dir.
\item func        : strs. Function and file to evaluate at each iteration of 
                          the MCMC.
                          Format: function file
                          Note: omit the '.py' from `file`.
\item pnames      : strs. Name of each free parameter. Can include LaTeX 
                          formatting.
\item pinit       : floats. Initial parameters for the MCMC.
\item pmin        : floats. Minima for free parameters.
\item pmax        : floats. Maxima for free parameters.
\item pstep       : floats. Step size for free parameters. 
                          This will change throughout the MCMC due to the 
                          differential evolution algorithm used.
\item niter       : int.  Number of total iterations.
\item burnin      : int.  Number of burned iterations from the beginning of 
                          chains.
\item nchains     : int.  Number of parallel samplers.
\item thinning    : int.  Thinning factor for posterior.
\item data        : floats. Values to be fit via MCMC. 
                    Format: Separate each value by an indented newline.
                    Alternatively, specify a .NPY file.
\item uncert      : floats. Uncertainties on values to be fit via MCMC. 
                          Same format (indented newlines or .NPY)
\item filters     : strs. Paths to filters associated with each datum. 
                          Same format (indented newlines only)
\item starspec    : str.  Path to .NPY file of the stellar spectrum.
\item factor      : str.  Path to .NPY file of multiplication factor by which 
                          to modify de-normalized predictions. 
                          E.g., unit conversion.
\item wnfact      : float. Multiplication factor to convert `xvals` to cm-1.
\item filt2um     : float. Multiplication factor to convert the filter 
                          wavelengths to microns.
\item PTargs      : str.  Path to .txt file containing values necessary to 
                          calculate the temperature--pressure profile.
                          Currently, only option is Line et al. (2013) method.
                          Format: R\_star (m), T\_star (K), T\_int (K), 
                                  sma (m), grav (cm s-2)

\end{itemize}

\noindent \underline{Plotting Parameters}
\begin{itemize}
\item xvals      : str.  Path to .NPY file containing the x-axis values 
                         associated with a prediction.
                         If relative, path is with respect to `inputdir`.
\item xlabel     : str.  X-axis label for plots.
\item ylabel     : str.  Y-axis label for plots.
\item fpress     : str.  Path to text file containing the pressures of each 
                         layer of the atmosphere, for plotting T(p) profiles.
                         If relative, with respect to `inputdir`.
\item savefile   : str.  (optional) Prefix for MCMC plots to be saved.
\end{itemize}



\section{Program Outputs}
\label{sec:outputs}

HOMER produces the following outputs:

\begin{itemize}
\item MCMC.log -- a record of the MCMC exploration, including SPEIS, ESS, and 
                  credible regions.
\item output.npy -- the posterior determined by the MCMC
\item MCMC plots -- pairwise, posterior, PT (if `plot\_PT`), and trace plots.
\item comparison plots -- if `compost`, makes plots of the 1D marginalized 
                          posteriors, 2D pairwise posteriors, and explored 
                          temperature--presure profiles.
\item bhatchar.npy -- if `compost`, calculates the Bhattacharyya coefficients 
                      between HOMER's posterior and the other specified 
                      posterior.
\end{itemize}



\section{Be Kind}
\label{sec:bekind}
Please cite this paper if you found this package useful for your
research:

\begin{itemize}
\item Himes et al. (2020), submitted to PSJ.
\end{itemize}

\begin{verbatim}
@article{HimesEtal2020psjMARGEHOMER,
   author = {{Himes}, Michael D. and {Harrington}, Joseph and {Cobb}, Adam D. and {G{\"u}ne{\textcommabelow s} Baydin}, At{\i}l{\i}m and {Soboczenski}, Frank and
         {O'Beirne}, Molly D. and {Zorzan}, Simone and
         {Wright}, David C. and {Scheffer}, Zacchaeus and
         {Domagal-Goldman}, Shawn D. and {Arney}, Giada N.},
    title = "Accurate Machine Learning Atmospheric Retrieval via a Neural Network Surrogate Model for Radiative Transfer",
  journal = {PSJ},
     year = 2020,
    pages = {submitted to PSJ}
}
\end{verbatim}

\noindent Thanks!

% \section{Further Reading}
% \label{sec:furtherreading}

% TBD: Add papers here.


\end{document}

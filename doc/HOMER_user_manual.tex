% HOMER User Manual
%
% Please note this document will be automatically compiled and hosted online
% after each commit to master. Because of this, renaming or moving the
% document should be done carefully. To see the compiled document, go to
% http://planets.ucf.edu/bart-docs/HOMER_user_manual.pdf

\documentclass[letterpaper, 12pt]{article}
\textwidth=6.5in
\textheight=9.5in
\topmargin=-0.75in
\oddsidemargin=0.0in
\evensidemargin=0.0in

\usepackage{graphicx}
\usepackage{enumitem}
\usepackage{amssymb, amsmath}
\usepackage{xcolor}
\usepackage{listings}

\usepackage{times}
\usepackage{natbib}
\usepackage{etoolbox}
\usepackage{astjnlabbrev-jh}
\usepackage{bibentry}
\usepackage{ifthen}
\usepackage{epsfig}

\usepackage{commath}
\usepackage{rotating}

\usepackage{dirtree}
\usepackage{changepage}
\usepackage{alltt}

%\usepackage[T1]{fontenc}
%\usepackage[scaled]{beramono}
%\usepackage[utf8]{inputenc}
%\renewcommand*\familydefault{\ttdefault}

% \lstset{
% language=Python,
% showstringspaces=false,
% formfeed=\newpage,
% tabsize=4,
% commentstyle=\itshape,
% basicstyle=\ttfamily,
% morekeywords={models, lambda, forms}
% }
 
% \newcommand{\code}[2]{
% \hrulefill
% \subsection*{#1}
% \lstinputlisting{#2}
% \vspace{2em}
% }

% Default fixed font does not support bold face
\DeclareFixedFont{\ttb}{T1}{txtt}{bx}{n}{10} % for bold
\DeclareFixedFont{\ttm}{T1}{txtt}{m}{n}{10}  % for normal
% twelve-sized ttm:
\DeclareFixedFont{\tttb}{T1}{txtt}{bx}{n}{12}  % for bold
\DeclareFixedFont{\tttm}{T1}{txtt}{m} {n}{12}  % for normal

\DeclareFixedFont{\ttnm}{T1}{txtt}{m}{n}{9.8}  % for normal

% Custom colors
\usepackage{color}
\definecolor{deepblue}  {rgb}{0.0, 0.0, 0.5}
\definecolor{deepred}   {rgb}{0.8, 0.0, 0.0}
\definecolor{deepgreen} {rgb}{0.0, 0.5, 0.0}
\definecolor{commentc}  {rgb}{0.5, 0.5, 0.5}
\definecolor{DodgerBlue}{rgb}{0.1, 0.6, 1.0}

% Python style for highlighting
\newcommand\pythonstyle{\lstset{
language=Python,
basicstyle = \ttm,
morekeywords = {self, as, assert, with, yield}, % Add keywords here
keywordstyle  = \ttb\color{blue},      %
emph        = {MyClass, __init__},     % Custom highlighting
emphstyle   = \ttb\color{DodgerBlue},  % Custom  highlighting style
stringstyle = \color{deepred},         % Strings highlighting style
commentstyle=\color{commentc},         % Comment highlighting style
frame       = tb,                      % Any extra options here
showstringspaces = false
}}

% Python environment:
\lstnewenvironment{python}[1][]{\pythonstyle\lstset{#1}}{}
% Python for external files:
\newcommand\pythonexternal[2][]{{\pythonstyle\lstinputlisting[#1]{#2}}}
% Python for inline:
\newcommand\pythoninline[1]{{\pythonstyle\lstinline!#1!}}

% Python style for highlighting
\newcommand\plainstyle{\lstset{
language=Python,
basicstyle = \ttnm,
keywordstyle  = \ttnm,      %
emph        = {MyClass, __init__},     % Custom highlighting
emphstyle   = \ttnm\color{black},    % Custom  highlighting style
stringstyle = \color{black},         % Strings highlighting style
commentstyle=\color{black},         % Comment highlighting style
frame       = tb,                      % Any extra options here
showstringspaces = false
}}

% Plain environment:
\lstnewenvironment{plain}[1][]{\plainstyle\lstset{#1}}{\vspace{10pt}}
\newcommand\plaininline[1]{{\plainstyle\lstinline!#1!}}

% To use boldface verbatim:
%\lstset{basicstyle=\ttfamily,
%        escapeinside={||},
%        mathescape=true}

\lstset{
    language={[LaTeX]TeX},
    basicstyle=\tt\color{red},
    escapeinside={||},
}

\bibliographystyle{apj_hyperref}
\usepackage[%pdftex,      %%% hyper-references for pdflatex
bookmarks=true,           %%% generate bookmarks ...
bookmarksnumbered=true,   %%% ... with numbers
colorlinks=true,          % links are colored
citecolor=blue,           % green   % color of cite links
linkcolor=blue,           %cyan,         % color of hyperref links
menucolor=blue,           % color of Acrobat Reader menu buttons
urlcolor=blue,            % color of page of \url{...}
breaklinks=true,
linkbordercolor={0 0 1},  %%% blue frames around links
pdfborder={0 0 1},
frenchlinks=true]{hyperref}
%\usepackage{breakurl}

\newcommand{\eprint}[1]{\href{http://arxiv.org/abs/#1}{#1}}
\newcommand{\ISBN}[1]{\href{http://cosmologist.info/ISBN/#1}{ISBN: #1}}
\providecommand{\adsurl}[1]{\href{#1}{ADS}}

% hyper ref only the year in citations:
\makeatletter
% Patch case where name and year are separated by aysep:
\patchcmd{\NAT@citex}
  {\@citea\NAT@hyper@{%
     \NAT@nmfmt{\NAT@nm}%
     \hyper@natlinkbreak{\NAT@aysep\NAT@spacechar}{\@citeb\@extra@b@citeb}%
     \NAT@date}}
  {\@citea\NAT@nmfmt{\NAT@nm}%
   \NAT@aysep\NAT@spacechar\NAT@hyper@{\NAT@date}}{}{}
% Patch case where name and year are separated by opening bracket:
\patchcmd{\NAT@citex}
  {\@citea\NAT@hyper@{%
     \NAT@nmfmt{\NAT@nm}%
     \hyper@natlinkbreak{\NAT@spacechar\NAT@@open\if*#1*\else#1\NAT@spacechar\fi}%
       {\@citeb\@extra@b@citeb}%
     \NAT@date}}
  {\@citea\NAT@nmfmt{\NAT@nm}%
   \NAT@spacechar\NAT@@open\if*#1*\else#1\NAT@spacechar\fi\NAT@hyper@{\NAT@date}}
  {}{}
\makeatother


%\def\bibAnnoteFile#1{}
%\bibpunct[, ]{(}{)}{,}{a}{}{,}

% Packed reference list:
\setlength\bibsep{0pt}

% \pagestyle{myheadings}
% \markright{MC\sp{3}}
% \pagenumbering{arabic}


% :::::::::::::::::::::::
\newcommand\degree{\degr}
\newcommand\degrees{\degree}
\newcommand\vs{\emph{vs.}}

% unslanted mu, for ``micro'' abbrev.
\DeclareSymbolFont{UPM}{U}{eur}{m}{n}
\DeclareMathSymbol{\umu}{0}{UPM}{"16}
\let\oldumu=\umu
\renewcommand\umu{\ifmmode\oldumu\else\math{\oldumu}\fi}
\newcommand\micro{\umu}
\newcommand\micron{\micro m}
\newcommand\microns{\micron}

\let\oldsim=\sim
\renewcommand\sim{\ifmmode\oldsim\else\math{\oldsim}\fi}
\let\oldpm=\pm
\renewcommand\pm{\ifmmode\oldpm\else\math{\oldpm}\fi}
\newcommand\by{\ifmmode\times\else\math{\times}\fi}
\newcommand\ttt[1]{10\sp{#1}}
\newcommand\tttt[1]{\by\ttt{#1}}
\newcommand\tablebox[1]{\begin{tabular}[t]{@{}l@{}}#1\end{tabular}}
\newbox{\wdbox}
\renewcommand\c{\setbox\wdbox=\hbox{,}\hspace{\wd\wdbox}}
\renewcommand\i{\setbox\wdbox=\hbox{i}\hspace{\wd\wdbox}}
\newcommand\n{\hspace{0.5em}}
\newcommand\marnote[1]{\marginpar{\raggedright\tiny\ttfamily\baselineskip=9pt #1}}
\newcommand\herenote[1]{{\bfseries #1}\typeout{======================> note on page \arabic{page} <====================}}
\newcommand\fillin{\herenote{fill in}}
\newcommand\fillref{\herenote{ref}}
\newcommand\findme[1]{\herenote{(FINDME: #1)}}

\newcount\timect
\newcount\hourct
\newcount\minct
\newcommand\now{\timect=\time \divide\timect by 60
         \hourct=\timect \multiply\hourct by 60
         \minct=\time \advance\minct by -\hourct
         \number\timect:\ifnum \minct < 10 0\fi\number\minct}

\newcommand\citeauthyear[1]{\citeauthor{#1} \citeyear{#1}}

\newcommand\mc{\multicolumn}
\newcommand\mctc{\multicolumn{2}{c}}


% {\tttm -h, --help} \\
% Print the list of arguments. \newline

% \newenvironment{myindentpar}[1]%
%   {\begin{list}{}%
%          {\setlength{\leftmargin}{3cm}}%
%      \item[]%
%   }
% {\end{list}}

\newenvironment{packed_enum}{
\begin{enumerate}[leftmargin=3cm]
   \setlength{\itemsep}{1pt}
   \setlength{\parskip}{5pt}
   \setlength{\parsep}{0pt}
}{\end{enumerate}}

\newcommand{\argument}[2]{{\noindent\tttm #1}%
\begin{adjustwidth}{2.5em}{0pt}%
#2 \vspace{0.3cm}%
\end{adjustwidth}%
}

%\newcommand{\ttmb}[1]{\tttm\color{#1}}
\newcommand{\routine}[2]{{\noindent\tttm\color{blue} #1:}%
\begin{adjustwidth}{2.0em}{0pt}%
#2 \vspace{0.15cm}%
\end{adjustwidth}%
}

% \newcommand{\routine}[2]{{\noindent\tttm\color{blue} #1}%
% \begin{adjustwidth}{0.0em}{0pt}%
%  #2 \vspace{0.15cm}%
% \end{adjustwidth}%
% }

% :::::::::::::::::: jhmacs2.tex :::::::::::::::::::::::::::::::::::::
\typeout{Joe Harrington's personal setup, Wed Jun 17 10:53:17 EDT 1998}
% Tue Mar 29 22:23:03 EST 1994

% :::::: pato.tex ::::::
% Joetex character unreservations.
% This file frees most of TeX's reserved characters, and provides
% several alternatives for their functions.


% utility
\catcode`@=11

% comments are first....
\newcommand\comment[1]{}

\newcommand\commenton{\catcode`\%=14}
\newcommand\commentoff{\catcode`\%=12}

% Not-a-comment:
\newcommand\nocomment[1]{#1}

\renewcommand\math[1]{$#1$}
\newcommand\mathshifton{\catcode`\$=3}
\newcommand\mathshiftoff{\catcode`\$=12}

\comment{alignment tab}
\let\atab=&
\newcommand\atabon{\catcode`\&=4}
\newcommand\ataboff{\catcode`\&=12}

\let\oldmsp=\sp
\let\oldmsb=\sb
\def\sp#1{\ifmmode
           \oldmsp{#1}%
         \else\strut\raise.85ex\hbox{\scriptsize #1}\fi}
\def\sb#1{\ifmmode
           \oldmsb{#1}%
         \else\strut\raise-.54ex\hbox{\scriptsize #1}\fi}
\newbox\@sp
\newbox\@sb
\def\sbp#1#2{\ifmmode%
           \oldmsb{#1}\oldmsp{#2}%
         \else
           \setbox\@sb=\hbox{\sb{#1}}%
           \setbox\@sp=\hbox{\sp{#2}}%
           \rlap{\copy\@sb}\copy\@sp
           \ifdim \wd\@sb >\wd\@sp
             \hskip -\wd\@sp \hskip \wd\@sb
           \fi
        \fi}
\def\msp#1{\ifmmode
           \oldmsp{#1}
         \else \math{\oldmsp{#1}}\fi}
\def\msb#1{\ifmmode
           \oldmsb{#1}
         \else \math{\oldmsb{#1}}\fi}
\def\supon{\catcode`\^=7}
\def\supoff{\catcode`\^=12}
\def\subon{\catcode`\_=8}
\def\suboff{\catcode`\_=12}
\def\supsubon{\supon \subon}
\def\supsuboff{\supoff \suboff}


\newcommand\actcharon{\catcode`\~=13}
\newcommand\actcharoff{\catcode`\~=12}

\newcommand\paramon{\catcode`\#=6}
\newcommand\paramoff{\catcode`\#=12}

\comment{And now to turn us totally on and off...}

\newcommand\reservedcharson{ \commenton  \mathshifton  \atabon  \supsubon
                             \actcharon  \paramon}

\newcommand\reservedcharsoff{\commentoff \mathshiftoff \ataboff \supsuboff
                             \actcharoff \paramoff}

\newcommand\nojoe[1]{\reservedcharson #1 \reservedcharsoff}

\catcode`@=12
\reservedcharsoff

\reservedcharson
\newcommand\jhauth[1]{{#1}}
\newcommand\jhstud[1]{{#1}}

\comment{Must have ONLY ONE of these... trust these macros, they work
\newcommand\jhauth[1]{{\bfseries #1}}
\newcommand\jhstud[1]{{\em #1}}
}

\reservedcharsoff
\reservedcharson


% ::::::::::::::::::::::::::::::::::::::::::::::::::::

\def\vs{{\em vs.}}
\def\p{\phantom{(0)}}

% Section levels:
\setcounter{secnumdepth}{5}
%  \section{}       % level 1
%  \subsection{}    % level 2
%  \subsubsection{} % level 3
%  \paragraph{}     % level 4 - equivalent to subsubsubsection
%  \subparagraph{}  % level 5

% To show in the table of content:
\setcounter{tocdepth}{5}

% Linebreak after \paragraph
\makeatletter
\renewcommand\paragraph{%
   \@startsection{paragraph}{4}{0mm}%
      {-\baselineskip}%
      {.5\baselineskip}%
      {\normalfont\normalsize\bfseries}}
\makeatother

% Linebreak after \paragraph
\makeatletter
\renewcommand\subparagraph{%
   \@startsection{subparagraph}{4}{0mm}%
      {-\baselineskip}%
      {.5\baselineskip}%
      {\normalfont\normalsize\bfseries}}
\makeatother

\actcharon
\renewcommand{\textfraction}{0.1}
\comment{\paramon\def\herenote#1{}\paramoff}
\renewcommand{\thepage}{\arabic{page}}
\reservedcharson

% :::::::::::: My Additions ::::::::::::::
\newcommand\Spitzer{{\em Spitzer}}
\newcommand\SST{{\em Spitzer Space Telescope}}
\newcommand\chisq{$\chi^2$}
\newcommand\itbf[1]{\textit{\textbf{#1}}}
\newcommand\bftt[1]{\texttt{\textbf{#1}}}
\newcommand\function[1]{\noindent\texttt{\begin{tabular}{@{}l@{}l}#1\end{tabular}}\newline}
\newcommand\bfv[1]{|\textbf{#1}|}
\newcommand\ttred[1]{\textcolor{red}{\ttfamily #1}}
\newcommand\ttblue[1]{\textcolor{blue}{\ttfamily #1}}
\newcommand\ttblack[1]{\textcolor{black}{\ttfamily #1}}
\newcommand\der{{\rm d}}
\newcommand\tno{$\sp{-1}$}
\newcommand\tnt{$\sp{-2}$}
\newcommand*\Eval[3]{\left.#1\right\rvert_{#2}^{#3}}
\newcommand\mcc{MC\sp{3}}

%:::::::::::::::::::::::::::::::::::::::::
% Next six lines adjust spacing above/below captions and Sections etc
% Adjust as needed

\comment{
% \setlength{\abovecaptionskip}{0pt}
% \setlength{\belowcaptionskip}{0pt}
% \setlength{\textfloatsep}{8pt}
% \titlespacing{\section}{0pt}{5pt}{*0}
% \titlespacing{\subsection}{0pt}{5pt}{*0}
% \titlespacing{\subsubsection}{0pt}{5pt}{*0}
}

\reservedcharsoff
\actcharon
\mathshifton

\reservedcharson


\begin{document}

\begin{titlepage}
\begin{center}

\textsc{\LARGE University of Central Florida}\\[1.5cm]

% Title
\rule{\linewidth}{0.5mm} \\[0.4cm]
{ \huge \bfseries MARGE Users Manual \\[0.4cm] }
\rule{\linewidth}{0.5mm} \\[1.0cm]

\textsc{\Large Helper Of My Eternal Retrievals}\\[1.5cm]

% Author and supervisor
\noindent
\begin{minipage}{0.4\textwidth}
\begin{flushleft}
\large
\emph{Authors:} \\
Michael \textsc{Himes} \\
\end{flushleft}
\end{minipage}%
\begin{minipage}{0.4\textwidth}
\begin{flushright} \large
\emph{Supervisor:} \\
Dr.~Joseph \textsc{Harrington}
\end{flushright}
\end{minipage}
\vfill

% Bottom of the page
{\large \today}

\end{center}
\end{titlepage}

\tableofcontents
\newpage

\section{Team Members}
\label{sec:team}

\begin{itemize}
\item \href{https://github.com/mdhimes/}{Michael Himes}%
  \footnote{https://github.com/mdhimes/}, University of
  Central Florida (mhimes@knights.ucf.edu)
\item Joseph Harrington, University of Central Florida
\item Zacchaeus Scheffer, University of Central Florida
\end{itemize}

\section{Introduction}
\label{sec:theory}

\noindent This document describes HOMER, the Helper Of My Eternal Retrievals.
HOMER retrieves atmospheric properties using a neural network (NN) model of 
radiative transfer (RT), such as that trained by MARGE.  HOMER uses the 
Multi-Core Markov Chain Monte Carlo (MC3) code to explore a parameter space 
and determine a posterior distribution of models consistent with the observed 
data.

\noindent The detailed HOMER code documentation and User Manual are provided 
with the package to assist users in its usage. For additional support, contact 
the lead author (see Section \ref{sec:team}).

\noindent HOMER is released under the Reproducible Research Software License.  
For details, see 
\url{https://planets.ucf.edu/resources/reproducible-research/software-license/}.

\noindent The HOMER package is organized as follows: \newline
% The framebox and minipage are necessary because dirtree kills the
% indentation.
\noindent\framebox{\begin{minipage}[t]{0.97\columnwidth}%
\dirtree{%
 .1 HOMER. 
 .2 code.
 .3 MCcubed.
 .4 MCcubed.
 .5 mc.
 .5 rednoise.
 .2 doc.
 .2 example.
 .2 lib. 
 .2 modules. 
 .3 MCcubed. 
}
\end{minipage}}
\vspace{0.7cm}
% \newline is not working here, therefore I use vspace.
% (because dirtree is such a pain in the ass)

\section{Installation}
\label{sec:installation}

\subsection{System Requirements}
\label{sec:requirements}

\noindent HOMER was developed on a Unix/Linux machine using the following 
versions of packages:

\begin{itemize}
\item Python 3.7.2
\item GPyOpt 1.2.5
\item Keras 2.2.4
\item Numpy 1.16.2
\item Matplotlib 3.0.2
\item Scipy 1.2.1
\item sklearn 0.20.2
\item Tensorflow 1.13.1
\item CUDA 9.1.85
\item cuDNN 7.5.00
\end{itemize}

This conveniently matches the dependencies of MARGE.



\subsection{Install and Compile}
\label{sec:install}

\noindent To begin, obtain the latest stable version of HOMER.  

\noindent First, create a local directory to hold HOMER.  Let the path to this directory 
be `localHOMERdir`.

\begin{verbatim}
mkdir `localHOMERdir`
cd `localHOMERdir`
\end{verbatim}

\noindent Now, clone the repository:
\begin{verbatim}
git clone --recursive https://github.com/exosports/HOMER .
\end{verbatim}

\noindent Now, modify MC3 as necessary:

\begin{verbatim}
make mccubed
\end{verbatim}

\noindent HOMER contains a file to easily build a conda environment capable of 
executing the software.  Create the environment via

\begin{verbatim}
conda env create -f environment.yml
\end{verbatim}

\noindent Then, activate the environment:

\begin{verbatim}
conda activate homer
\end{verbatim}

\noindent You are now ready to run HOMER.


\section{Example}
\label{sec:example}

\noindent The following script will walk a user through executing all modes of MARGE 
to simulate the emission spectra of HD 189733 b with a variety of thermal 
profiles and atmospheric compositions, process the data, and train an NN 
model to quickly approximate spectra.  These instructions are meant to be 
executed from a Linux terminal.  Note that the complete execution of all steps 
requires significant compute resources, especially if lacking a graphics 
processing unit (GPU).

The following script will walk a user through using HOMER for a retrieval on 
HD 189733 b, following from MARGE's example which trains an NN model for RT 
for this planet.

\noindent To begin, copy the requisite files to a directory parallel to HOMER.  From 
the HOMER directory `localHOMERdir`,

\begin{verbatim}
mkdir ../run
cp -a ./example/* ../run/.
cd ../run
\end{verbatim}

\noindent Now, execute HOMER:

\begin{verbatim}
../`localHOMERdir`/HOMER.py example.cfg
\end{verbatim}

\noindent This will take some time to run. 


\section{Program Inputs}
\label{sec:inputs}

The executable HOMER.py is the driver for the HOMER program. It takes a 
a configuration file of parameters.  Once configured, HOMER is executed via 
the terminal as described in Section \ref{sec:example}.


\subsection{HOMER Configuration File}
\label{sec:config}
The HOMER configuration file is the main file that sets the arguments for a 
HOMER run. The arguments follow the format {\ttb argument = value}, where 
{\ttb argument} is any of the possible arguments described below. 

\noindent The available options for a HOMER configuration file are listed below.

\noindent \underline(Directories)
\begin{enumerate}
\item inputdir   : str.  Directory containing HOMER inputs.
\item outputdir  : str.  Directory containing HOMER outputs.
\end{enumerate}

\noindent \underline{Run Parameters}
\begin{enumerate}
\item onlyplot    : bool. Determines whether to skip the MCMC.
                    Reproduces plots & calculations related to posterior.
\item credregion  : bool. Determines whether to calculate the 68, 95, and 99\% 
                    credible regions & uncertainties.
\item compost     : bool. Determines whether to compare HOMER's posterior to 
                          another.
\item compfile    : str.  Path to posterior to compare with HOMER.
\item compname    : str.  Name of the other posterior for plot legends.
\item compsave    : str.  File name prefix for the saved comparison plots.
\item compshift   : floats. Shifts all values of a particular parameter by a 
                          set amount in the posterior to be compared, such as 
                          for unit conversions.
                    Format: val1 val2 val3 val4 ...
\item postshift   : floats. Same as `compshift`, but for HOMER's posterior.
\end{enumerate}


\noindent \underline{Data Normalization Parameters}
\begin{enumerate}
\item ilog        : bool. Determines whether the NN takes the logarithm of the 
                          inputs.
\item olog        : bool. Determines whether the NN predicts the log of the 
                          outputs.
\item normalize   : bool. Determines whether to standardize the data by its 
                          mean and standard deviation.
\item scale       : bool. Determines whether to scale the data to be within a 
                          range.
\item scalelims   : ints. Range to scale the data to.
                          Format: low, high
\item fmean       : str.  Path to .NPY file of mean training input and output 
                          values.
                          Format: [inp0, inp1, ..., outp0, outp1, ...]
                          If relative path, assumed to be with respect to the 
                          input directory.
\item fstdev      : str.  Path to .NPY file of standard deviation of 
                          inputs/outputs.
                          See `fmean` for format & path description.
\item fmin        : str.  Path to .NPY file of minima of inputs/outputs.
                          See `fmean` for format & path description.
\item fmax        : str.  Path to .NPY file of maxima of inputs/outputs.
                          See `fmean` for format & path description.
\end{enumerate}


\noindent \underline{Neural Network (NN) Parameters}
\begin{enumerate}
\item weight_file: str.  File containing NN model weights.
                   NOTE: MUST end in .h5
\item input_dim  : int.  Dimensionality of the input  to the NN.
\item output_dim : int.  Dimensionality of the output of the NN.
\item convlayers : ints. Number of nodes for each convolutional layer.
\item denselayers: ints. Number of nodes for each dense         layer.
\end{enumerate}

\noindent \underline{MCMC Parameters}
\begin{enumerate}
\item flog        : str.  Path to MCMC log file. 
                          If relative, with respect to input dir.
\item func        : strs. Function and file to evaluate at each iteration of 
                          the MCMC.
                          Format: function file
                          Note: omit the '.py' from `file`.
\item pnames      : strs. Name of each free parameter. Can include LaTeX 
                          formatting.
\item pinit       : floats. Initial parameters for the MCMC.
\item pmin        : floats. Minima for free parameters.
\item pmax        : floats. Maxima for free parameters.
\item pstep       : floats. Step size for free parameters. 
                          This will change throughout the MCMC due to the 
                          differential evolution algorithm used.
\item niter       : int.  Number of total iterations.
\item burnin      : int.  Number of burned iterations from the beginning of 
                          chains.
\item nchains     : int.  Number of parallel samplers.
\item thinning    : int.  Thinning factor for posterior.
\item data        : floats. Values to be fit via MCMC. 
                          Format: Separate each value by an indented new line.
\item uncert      : floats. Uncertainties on values to be fit via MCMC. 
                          Same format.
\item filters     : strs. Paths to filters associated with each datum. 
                          Same format.
\item starspec    : str.  Path to .NPY file of the stellar spectrum.
\item factor      : str.  Path to .NPY file of multiplication factor by which 
                          to modify de-normalized predictions. 
                          E.g., unit conversion.
\item wnfact      : float. Multiplication factor to convert `xvals` to cm-1.
\item filt2um     : float. Multiplication factor to convert the filter 
                          wavelengths to microns.
\item PTargs      : str.  Path to .txt file containing values necessary to 
                          calculate the temperature--pressure profile.
                          Currently, only option is Line et al. (2013) method.
                          Format: R_star (m), T_star (K), T_int (K), 
                                  sma    (m), grav (cm s-2)

\end{enumerate}

\noindent \underline{Plotting Parameters}
\begin{enumerate}
\item xvals      : str.  Path to .NPY file containing the x-axis values 
                         associated with a prediction.
                         If relative, path is with respect to `inputdir`.
\item xval_label : str.  X-axis label for plots.
\item yval_label : str.  Y-axis label for plots.
\item fpress     : str.  Path to text file containing the pressures of each 
                         layer of the atmosphere, for plotting T(p) profiles.
                         If relative, with respect to `inputdir`.
\item savefile   : str.  Prefix for MCMC plots to be saved.
\end{enumerate}



\section{Program Outputs}
\label{sec:outputs}

HOMER produces the following outputs:

\begin{enumerate}
\item MCMC.log -- a record of the MCMC
\item output.npy -- the posterior determined by the MCMC
\item MCMC plots -- pairwise, posterior, PT, and trace plots.
\item ess.txt -- the effective sample size (ESS) of the run.
\item credregion.txt -- the 68, 95, and 99\% credible regions.
\item comparison plots -- if `compost`, makes plots of the 1D marginalized 
                          posteriors, 2D pairwise posteriors, and explored 
                          temperature--presure profiles.
\item bhatchar.npy -- if `compost`, calculates the Bhattacharyya coefficients 
                      between HOMER's posterior and the other specified 
                      posterior.
\end{enumerate}



\section{Be Kind}
\label{sec:bekind}
Please cite this paper if you found this package useful for your
research:

\begin{enumerate}
\item Himes et al. (2020), submitted to PSJ.
\end{enumerate}

\begin{verbatim}
@article{HimesEtal2020psjMARGEHOMER,
   author = {{Himes}, Michael D. and {Harrington}, Joseph and {Cobb}, Adam D. and {G{\"u}ne{\textcommabelow s} Baydin}, At{\i}l{\i}m and {Soboczenski}, Frank and
         {O'Beirne}, Molly D. and {Zorzan}, Simone and
         {Wright}, David C. and {Scheffer}, Zacchaeus and
         {Domagal-Goldman}, Shawn D. and {Arney}, Giada N.},
    title = "Accurate Machine Learning Atmospheric Retrieval via a Neural Network Surrogate Model for Radiative Transfer",
  journal = {PSJ},
     year = 2020,
    pages = {submitted to PSJ}
}
\end{verbatim}

\noindent Thanks!

% \section{Further Reading}
% \label{sec:furtherreading}

% TBD: Add papers here.


\end{document}
